\documentclass[letterpaper]{article} % DO NOT CHANGE THIS
\usepackage[submission]{aaai25}  % DO NOT CHANGE THIS
\usepackage{times}  % DO NOT CHANGE THIS
\usepackage{helvet}  % DO NOT CHANGE THIS
\usepackage{courier}  % DO NOT CHANGE THIS
\usepackage[hyphens]{url}  % DO NOT CHANGE THIS
\usepackage{graphicx} % DO NOT CHANGE THIS
\urlstyle{rm} % DO NOT CHANGE THIS
\def\UrlFont{\rm}  % DO NOT CHANGE THIS
\usepackage{natbib}  % DO NOT CHANGE THIS AND DO NOT ADD ANY OPTIONS TO IT
\usepackage{caption} % DO NOT CHANGE THIS AND DO NOT ADD ANY OPTIONS TO IT
\usepackage{booktabs} %TODO check if allowable
\usepackage{microtype}
\usepackage{enumitem}
\usepackage{amsfonts,amsmath,amssymb}
\usepackage{bm}
\frenchspacing  % DO NOT CHANGE THIS
\setlength{\pdfpagewidth}{8.5in} % DO NOT CHANGE THIS
\setlength{\pdfpageheight}{11in} % DO NOT CHANGE THIS
%
\usepackage{nicematrix}

% These are recommended to typeset algorithms but not required. See the subsubsection on algorithms. Remove them if you don't have algorithms in your paper.
\usepackage{algorithm}
\usepackage[noend]{algpseudocode}

\usepackage{xcolor}         % colors
\usepackage{amsmath}
\usepackage{amssymb}

%ODO confirmed this is allowed by AAAI
\DeclareMathOperator*{\argmax}{argmax}
%TODO confirm this is allowed by AAAI
\usepackage{subcaption}

%
% These are are recommended to typeset listings but not required. See the subsubsection on listing. Remove this block if you don't have listings in your paper.
\usepackage{newfloat}
\usepackage{listings}
\DeclareCaptionStyle{ruled}{labelfont=normalfont,labelsep=colon,strut=off} % DO NOT CHANGE THIS
\lstset{%
	basicstyle={\footnotesize\ttfamily},% footnotesize acceptable for monospace
	numbers=left,numberstyle=\footnotesize,xleftmargin=2em,% show line numbers, remove this entire line if you don't want the numbers.
	aboveskip=0pt,belowskip=0pt,%
	showstringspaces=false,tabsize=2,breaklines=true}
\floatstyle{ruled}
\newfloat{listing}{tb}{lst}{}
\floatname{listing}{Listing}
%
% Keep the \pdfinfo as shown here. There's no need
% for you to add the /Title and /Author tags.
\pdfinfo{
/TemplateVersion (2025.1)
}

% DISALLOWED PACKAGES
% \usepackage{authblk} -- This package is specifically forbidden
% \usepackage{balance} -- This package is specifically forbidden
% \usepackage{color (if used in text)
% \usepackage{CJK} -- This package is specifically forbidden
% \usepackage{float} -- This package is specifically forbidden
% \usepackage{flushend} -- This package is specifically forbidden
% \usepackage{fontenc} -- This package is specifically forbidden
% \usepackage{fullpage} -- This package is specifically forbidden
% \usepackage{geometry} -- This package is specifically forbidden
% \usepackage{grffile} -- This package is specifically forbidden
% \usepackage{hyperref} -- This package is specifically forbidden
% \usepackage{navigator} -- This package is specifically forbidden
% (or any other package that embeds links such as navigator or hyperref)
% \indentfirst} -- This package is specifically forbidden
% \layout} -- This package is specifically forbidden
% \multicol} -- This package is specifically forbidden
% \nameref} -- This package is specifically forbidden
% \usepackage{savetrees} -- This package is specifically forbidden
% \usepackage{setspace} -- This package is specifically forbidden
% \usepackage{stfloats} -- This package is specifically forbidden
% \usepackage{tabu} -- This package is specifically forbidden
% \usepackage{titlesec} -- This package is specifically forbidden
% \usepackage{tocbibind} -- This package is specifically forbidden
% \usepackage{ulem} -- This package is specifically forbidden
% \usepackage{wrapfig} -- This package is specifically forbidden
% DISALLOWED COMMANDS
% \nocopyright -- Your paper will not be published if you use this command
% \addtolength -- This command may not be used
% \balance -- This command may not be used
% \baselinestretch -- Your paper will not be published if you use this command
% \clearpage -- No page breaks of any kind may be used for the final version of your paper
% \columnsep -- This command may not be used
% \newpage -- No page breaks of any kind may be used for the final version of your paper
% \pagebreak -- No page breaks of any kind may be used for the final version of your paperr
% \pagestyle -- This command may not be used
% \tiny -- This is not an acceptable font size.
% \vspace{- -- No negative value may be used in proximity of a caption, figure, table, section, subsection, subsubsection, or reference
% \vskip{- -- No negative value may be used to alter spacing above or below a caption, figure, table, section, subsection, subsubsection, or reference

% THEOREMS -------------------------------------------------------
\newtheorem{thm}{Theorem}
\newtheorem{cor}[thm]{Corollary}
\newtheorem{lem}{Lemma}
\newtheorem{prop}[thm]{Proposition}
\newtheorem{defn}{Definition}
\newtheorem{rem}{Remark}
\newtheorem{ex}{Example}


\newcommand{\tao}{\textsf{TaO-MG}}
\newcommand{\mohito}{\textsc{Mohito}}
\newcommand{\pgella}{\textsc{TaO-PGELLA}}
\newcommand{\rideshare}{\textsf{Rideshare}}
\newcommand{\wildfire}{\textsf{Wildfire Suppression}}
\newcommand*{\commt}[1]{\color{blue}\em{#1}}



\setcounter{secnumdepth}{2} %May be changed to 1 or 2 if section numbers are desired.

% The file aaai25.sty is the style file for AAAI Press
% proceedings, working notes, and technical reports.
%

% Title

% Your title must be in mixed case, not sentence case.
% That means all verbs (including short verbs like be, is, using,and go),
% nouns, adverbs, adjectives should be capitalized, including both words in hyphenated terms, while
% articles, conjunctions, and prepositions are lower case unless they
% directly follow a colon or long dash
\title{MOHITO: Multi-Agent Reinforcement Learning using Hypergraphs for Task-Open Systems}
\author{
    %Authors
    % All authors must be in the same font size and format.
    Written by AAAI Press Staff\textsuperscript{\rm 1}\thanks{With help from the AAAI Publications Committee.}\\
    AAAI Style Contributions by Pater Patel Schneider,
    Sunil Issar,\\
    J. Scott Penberthy,
    George Ferguson,
    Hans Guesgen,
    Francisco Cruz\equalcontrib,
    Marc Pujol-Gonzalez\equalcontrib
}
\affiliations{
    %Afiliations
    \textsuperscript{\rm 1}Association for the Advancement of Artificial Intelligence\\
    % If you have multiple authors and multiple affiliations
    % use superscripts in text and roman font to identify them.
    % For example,

    % Sunil Issar\textsuperscript{\rm 2},
    % J. Scott Penberthy\textsuperscript{\rm 3},
    % George Ferguson\textsuperscript{\rm 4},
    % Hans Guesgen\textsuperscript{\rm 5}
    % Note that the comma should be placed after the superscript

    1101 Pennsylvania Ave, NW Suite 300\\
    Washington, DC 20004 USA\\
    % email address must be in roman text type, not monospace or sans serif
    proceedings-questions@aaai.org
%
% See more examples next
}

%Example, Single Author, ->> remove \iffalse,\fi and place them surrounding AAAI title to use it
\iffalse
\title{My Publication Title --- Single Author}
\author {
    Author Name
}
\affiliations{
    Affiliation\\
    Affiliation Line 2\\
    name@example.com
}
\fi

\iffalse
%Example, Multiple Authors, ->> remove \iffalse,\fi and place them surrounding AAAI title to use it
\title{\mohito{}: Multi-Agent Reinforcement Learning using Hypergraphs for Task-Open Systems}
\author {
    % Authors
    First Author Name\textsuperscript{\rm 1},
    Second Author Name\textsuperscript{\rm 2},
    Third Author Name\textsuperscript{\rm 1}
}
\affiliations {
    % Affiliations
    \textsuperscript{\rm 1}Affiliation 1\\
    \textsuperscript{\rm 2}Affiliation 2\\
    firstAuthor@affiliation1.com, secondAuthor@affilation2.com, thirdAuthor@affiliation1.com
}
\fi

\begin{document}

\maketitle



The purpose of this proof is to show three things:
\begin{itemize}
    \item The space complexity of the interaction/observation graphs.
    \item The space complexity of \mohito{} \textbf{training} and \textbf{execution}.
    \item The time complexity of \mohito{} \textbf{training} and \textbf{execution}.
\end{itemize}



\begin{equation}
L_{\pi_i}=\sum_{j \in \text{sample batch}} -Q_i^{\bm{\phi}}(G^{j}, ed^j) + \lambda_A |\bm{\theta}_i - \bm{\theta}'_i|. %+ 
    \label{eq:actor_pg}
\end{equation}


\begin{align}
        L_{Q_i}= & \frac{1}{S}\sum_{j\in \text{batch}} \left ( r_i^j + \gamma Q_i^{\bm{\phi}'}(G'^j, ed'^j) - Q_i^{\phi}(G^{j},ed^j) \right )^2 \nonumber\\ 
        & + \lambda_C |\bm{\phi} - \bm{\phi}'|
    \label{eq:critic-loss}
\end{align}



\begin{algorithm}[!ht]
\caption{\mohito
% \\After collecting samples into a batch (lines 4-9), \mohito{} utilizes the batch to engage in actor-critic training (lines 12-20) using actor loss (Eq. \ref{eq:actor_pg}) and critic loss (Eq. \ref{eq:critic-loss}).
}
\begin{small}
\begin{algorithmic}[1]
\For{ $episode \gets 1$ to $N$}
    \State Get observation graphs $S = (S_1, S_2, \ldots, S_{|Ag|})$ from state with current tasks $X$ %$\{\mathcal{T}_0, \ldots ,\mathcal{T}_t\}$
    \While{ $episode$ not terminated}
    \For{each actor $i$}
        \State $a_i, ed_i \gets \pi^{\bm{\theta}_i}(S_i)$ 
    \EndFor
    \State Perform joint $(a_1, a_2, \ldots, a_{|Ag|})$ with prob. (1 - $\epsilon$) else a random joint action. Get 
    next state $S'$ and reward $r$ 
    \For{each actor $i$}
        \State $a'_i, ed'_i \gets \pi^{\bm{\theta'}_i}(S_i')$ 
    \EndFor
    \State $G, G' \gets \text{\em{generateCriticGraph}}(S, S')$
    \State $\text{batch} \gets \text{batch}\cup (G, {\bf ed}, r, G', {\bf ed'})$
    
        \If{$size($batch$) = B$}
            %CRITIC UPDATE
            \State Compute loss $L_{Q_i}$ (Eq.~\ref{eq:critic-loss}) for agent $i$
            \State Backpropagate $L_{Q_i}$ and update $Q^{\bm{\phi}_i}$~~~ $\forall i$
            %ACTOR UPDATE
            \State Compute loss $L_{\pi_i}$ (Eq.~\ref{eq:actor_pg}) for agent $i$
            \State Backpropagate $L_{\pi_i}$ and update $\pi^{\bm{\theta}_i}$~~~ $\forall i$
            \State Clear batch
        \EndIf
     \EndWhile
        \For{each agent $i$ after every $K$ episodes}
            % \State Slow update 
            \State $ \bm{\theta}'_i  \gets \psi_A \times \bm{\theta}_{i} + (1-\psi_A) \times \bm{\theta}'_{i}$
            \State $ \bm{\phi}'_i  \gets \psi_Q \times \bm{\phi}_{i} + (1-\bm{\psi}_Q) \times \bm{\phi}'_{i}$
        \EndFor
        \State $S \gets S'$
\EndFor
\end{algorithmic}
\end{small}
\label{alg:MOHITO}
\end{algorithm}

\section{Appendix ? complexity analysis}

Here we evaluate the space and time complexity of our hypergraphs and \mohito{}. To do this we consider the following as given variables: tasks, $X$; actions, $A$; agents, $Ag$; hyperedges, $E$; batch size.

\subsection{Space complexity}

\subsubsection{Interaction and observation graphs}

Consider an arbitrary graph $g$. The space used to store $g$ is the sum of the space used to store $g$'s edges, $g_e$, and $g$'s nodes, $g_n$. We use Eq:\ref{eq:graphsize} to find $|g|$.

\begin{equation}
    \label{eq:graphsize}
    \begin{split}
    |g| =& (|g_n| \times |features|)\\+& (2 \times|g_e| + |g_e| \times |edge\_features|)
\end{split}
\end{equation}

\textit{Observation} graphs, $S_i\forall i \in Ag$, store public and $i$'s private observations of tasks $X$; actions, $A$; agents, $Ag$; hyperedges, $E$, see Eq:\ref{eq:graphnodesize}. Each of these groups are nodes with features,  $|n|\forall n \in S^n_i \leq |x| \forall x \in X$ . Observation graphs are used by the actor. Meanwhile the critic uses a \textit{interaction} graph, $G,$ where $ S_i\subseteq G \forall i \in Ag$. 

\begin{equation}
\label{eq:graphnodesize}
    |G_n| = |A| + |Ag| + |X| + |E|
\end{equation}



 The only edges in the observation graph connect to $E$, $<(ag,e), (x,e), (a,e)>$. These represent agent action spaces, $A_i=\bigcup_{x\in X}A_{x,i}$. There are no other edges, so $|G_e|=3\times |E|$.

 We substitute the node and edge spaces into Eq:\ref{eq:graphsize} to get Eq:\ref{eq:interactiongraph}. Recall that $S_i \subseteq G$, so $|S_i|=O(|G|)$. 

 \begin{equation}
 \label{eq:interactiongraph}
 \begin{split}
      |G|& \leq |x| \times (|A| + |Ag| + |X| + |E|) + (6\times |E| + 0)\\
     &|G| = O\big(|x| \times(|A| + |Ag| + |X| + |E|) + |E|\big)
 \end{split}
 \end{equation}

 \subsubsection{\mohito{} execution}

 \mohito{} actors are comprised of $n$-many graph attention transformers (GAT) \cite{GAT} layers, activated by ReLU layers, and the hyperedge, $ed$, is selected by ArgMax. GAT work doesn't show space complexity, but they state it can scale linearly in $|G_n|$ and $|G_e|$ \cite{GAT}. Here we assume they scale polynomially in input and output feature size, $f,f'$, following their defined single head GAT time complexity Eq:\ref{eq:gat-time-complexity}. 

\begin{equation}
\label{eq:gat-time-complexity}
    GAT= O(|G_n|\times f \times f' + |G_e| * f')
\end{equation}

Within \mohito{}, $f = \{|x|,\text{hidden dim}\}, f'= \{|x|,\text{hidden dim}, \text{hidden dim }\times \text{heads}\}$. Often hidden dim $\times$ heads $> |x|$, but we consider these parameters as constants for this analysis. We substitute $|G_e|$, $|G_n|$, and simplify with Eq:\ref{eq:gat-time-complexity} to get \mohito{} ($M$) execution space complexity Eq:\ref{eq:mohito-execution-complexity}. Here $M^{space}_{exe}=actor_{space}$ because we are assuming no threading.

\begin{equation}
\small
\label{eq:mohito-execution-complexity}
\begin{split}
M_{exe}^{space} \leq& |x|^2 \times (|A| + |Ag| + |X| + |E|) + 3\times|E|\times |x|\\
&M_{exe}^{space}=O(|x| \times |G|)
\end{split}
\end{equation}


\subsubsection{Critic loss, $L_{Q}$}

%TODO unclear if this is incorporated into the described GAT space complexity from the GAT paper. I need to just look at the source code. 
The distinction between the critic and actor networks, linear layers and global mean pooling are both $O(|x| \times |Ag|)$ and strictly dominated by $|G|$ because $|x|\in \mathbb{N}$. This dominance also handles the complexity of storing $[ed,...ed_n]$, so we can say $critic_{space} = O(actor_{space})$. 
%Typically $\phi > \theta$, but when only considering a given graph as variable $\phi,\theta = O(|x|^m)$ for some constant $m$. \textcolor{red}{For now ignoring this.}

Here we consider spatial growth during $L_Q$. We perform two passes $Q_i^\phi(j), Q_i^{\phi'}(j)$ per agent per experience in the batch. The size of output Q-values is 1, and $|r^j_i|=1$. The space complexity of this loss is dominated by $critic_{space}$ and the batched inputs. 
%TODO put some other sentence here referencing the eq
% These dominants reveal the main source of growth is the GAT space complexity as we see comparing Eq:\ref{eq:critic-loss} with Eq:\ref{eq:gat-time-complexity} and the batch size. 

\begin{equation}
\small
\label{eq:critic-loss}
    M^{space}_{L_Q} = O\Big( M^{space}_{exe} + \text{batch size} \times \big( |G|+ (|Ag|\times |x|)\big)\Big)
\end{equation}


 

\subsubsection{Actor loss, $L_{\pi}$}

As with $L_Q$ the regularization only takes a multiple of parameter space, $\theta$ additional space here. Here we calculate $Q_i^{'\phi}(j)$, or in other words, using the updated main network $\phi$. The lack of need for $ed^{'j},G^{'j}$ in this loss function are multiplicative changes, and do not impact the space complexity, \ref{eq:policy-loss}.

\begin{equation}
\label{eq:policy-loss}
M^{space}_{L_\pi} = O(M^{space}_{L_Q})
\end{equation}


\subsection{Time complexity}


\bibliography{oasys}

\end{document}

